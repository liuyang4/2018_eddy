Hydroelectricity is a source of renewable energy with numerous favourable features: safe, clean, highly efficient, low-cost, and flexible. However, the deregulation of electricity markets and the integration of wind power brought important changes to the operating patterns of hydro units. Hydro turbines have to operate at off-design conditions to adapt to the demands of the grid. At off-design operating conditions, the reliability and safety of the hydro turbines are challenged by the unsteady complex flow. Accurate numerical simulations with validation against experiments must be conducted to better understand the flow phenomenon and thus prevent the potential risks.
%%%%%%%%%%%%%%%%%%%%%%%%%%%%%%%%%%%%%%%%%%%%%%%%%%%%

%%%%%%%%%%%%%%%%%%%%%%%%%%%%%%%%%%%%%%%%%%%%%%%%%%%%
The draft tube is an important component of a hydro turbine. The role of the draft tube is to decelerate the flow from the turbine runner and convert its dynamic pressure into a rise of static pressure \cite{sick2002cfd}. At part load conditions, the energy is not fully utilized by the runner-shaft-rotor, resulting in high residual swirl when the flow exits the runner. The swirling flow with high angular momentum enters the draft tube and encounters the lower momentum fluid in the inlet conical zone. The shear layer between them gives rise to a large helical precessing vortex, commonly referred to as a vortex rope. Due to the adverse pressure gradient and hydraulic instabilities present in the draft tube, the vortex breaks down and a reverse flow occurs, further complicating the flow phenomena. As a consequence, at part load conditions, the turbine experiences severe low frequency and large amplitude pressure fluctuations induced by the vortex rope in the draft tube. The pressure fluctuations will not only cause variations in the power output, but also endangers other hydro turbine components, especially when its frequency approaches the natural frequency of the turbine \cite{dorfler2012flow}. Thus, knowledge of the dynamic load on the turbine is a major concern of the hydraulic community, and the simulation of flows in draft tubes has attracted attention during the past two decades.
%%%%%%%%%%%%%%%%%%%%%%%%%%%%%%%%%%%%%%%%%%%%%%%%%%%%%
%%%%%%%%%%%%%%%%%%%%%%%%%%%%%%%%%%%%%%%%%%%%%%%%%%%%%
In order to accurately capture the flow features, efforts have been made by employing more advanced turbulence models, improving the grid quality and extending the computational domain to the complete hydro turbine. Guidelines to choose an appropriate turbulence model often begins from past numerical simulations and {\it a priori} knowledge of inherent flow features present in the flow. Presently, access to large computational resources have enabled the research community to employ increasingly resource-demanding models; from one-equation Reynolds-averaged Navier-Stokes (RANS) closure models to detached eddy simulation (DES) and large eddy simulation (LES).
%%%%%%%%%%%%%%%%%%%%%%%%%%%%%%%%%%%%%%%%%%%%%%%%%%%%%
%%%%%%%%%%%%%%%%%%%%%%%%%%%%%%%%%%%%%%%%%%%%%%%%%%%%%
Early efforts to solve the RANS equations with the standard k-$\epsilon$ turbulence model \cite{sick2002cfd, ruprecht2002simulation} were found to be over diffusive for the simulation of draft tube flows. 
%However, the inclusion of the runner \cite{ciocan2007experimental} provided the means to resolve the vortex rope frequency but at a less desirable radius and the impact on the general performance of the draft tube such as head losses and dynamic loads were not investigated.  
Various alternative turbulence models for RANS have been applied, including the Reynolds Stress Model (RSM) \cite{sick2002cfd, stein2006numerical, jovst2009numerical, jovst2011numerical} and the k-$\omega$ SST turbulence model \cite{foroutan2012simulation, foroutan2014flow, krappel2014investigation}; however, there is no established consensus on which turbulence closure model is suitable to capture the complex flow field. Unlike RANS, in which all turbulence scales are modelled, LES resolves the large eddies and models the effect of small-scale turbulence through the use of subgrid-scale (SGS) models. Early LES attempts \cite{chen1995multi,skotak2000helical,guo2006large} employed coarse computational grids in the order of one to three million points that today would be classified as severely under resolved LES. Since the Reynolds number in the draft tube is on the level of $10^{6}$, and the estimated grid-point requirement for wall-resolved LES is proportional to $Re_{L_{x}}^{13/7}$, then the computational domain should contain approximately $1.39\times 10^{11}$ grid points. As expected early results \cite{chen1995multi,skotak2000helical} did not improve upon prevailing RANS simulations but Guo et al. \cite{guo2006large} captured the draft tube vortex rope frequency, resolved the axial and tangential velocity profiles downstream of the inlet, and acquired a reasonable agreement on the pressure pulsation against experimental data. The results may have been due to the use of accurate boundary conditions through the inclusion of the runner, then the accurate modelling of the turbulent flow field through the use of LES. This observation was further collaborated through an examination \cite{jovst2011numerical} of both a 23.5 million grid point LES and a 5.3 million grid point Unsteady RANS (URANS) simulation, where only the URANS simulation included a runner which, surprisingly yielded a more resolved vortex rope and frequency. Except neither produced an accurate measure of the pressure pulsation in the draft tube. %Since the inclusion of the runner for wall-resolved draft tube LES simulations would be presently intractable, an hybrid RANS/LES or DES approach that employs a turbulence closure model for the RANS region and a SGS model for the LES region provides a promising avenue. DES simulations \cite{jovst2011numerical, krappel2014investigation} have been able to present the vortex rope on a wide range of test cases with various grid configurations and sizes, and flow solvers, with greater visual evidence of finer turbulent structures and well defined vortex ropes. Even so, the improved flow features have not translated to an improvement against experimental measurement of global quantities \cite{jovst2011numerical}.
%%%%%%%%%%%%%%%%%%%%%%%%%%%%%%%%%%%%%%%%%%%%%%%%%%%%%
%%%%%%%%%%%%%%%%%%%%%%%%%%%%%%%%%%%%%%%%%%%%%%%%%%%%%

Since turbulence models are not the only determinant factor of numerical simulations, a thorough investigation should include the quality of the numerical grids and the role of numerical schemes. Initial grid studies \cite{stein2006numerical, krappel2014investigation} demonstrated an enhanced amount of turbulent structures on finer grids and in \cite{krappel2014investigation} the turbulent kinetic energy spectra showed a well represented inertial range of turbulence at both the draft tube cone and diffuser. %Despite this, the investigations lacked a systematic approach typically expected for a grid convergence study and the impact on global parameters such as head losses and dynamic loads remains unresolved. 
Magnan et al. \cite{magnan2014challenges} conducted the most extensive grid convergence study to date and assessed the impact on several global parameters and concluded that poor grid quality severely impacts the accuracy of turbulence models, where disparate averaged axial velocities and turbulent kinetic energy contours were observed. %This warrants a comprehensive investigation on the sensitivity of the resolution of the vortex rope, accurate turbulent statistics, and most importantly global parameters on the grid density and distribution within the draft tube. 
%%%%%%%%%%%%%%%%%%%%%%%%%%%%%%%%%%%%%%%%%%%%%%%%%%%%%


%%%%%%%%%%%%%%%%%%%%%%%%%%%%%%%%%%%%%%%%%%%%%%%%%%%%%
Apart from turbulence models and grid effects, discretization schemes have a tremendous impact but their role has not been carefully examined in the context of draft tube flows. The proper resolution of fine turbulent structures not only require an appropriate time-step and grid spacing but a low dissipative scheme. This can be achieved through the use of high-order methods. Higher than second-order methods are currently intractable for draft tube flows due to the large Reynolds number and computational grid, but there is room for improvement in the development of novel low dissipative second-order schemes for vortex dominated or shear flow driven flows. Current second-order schemes employ reconstructions of the left and right states of the numerical fluxes to ensure the scheme remains second-order in smooth regions of the flow. However, variable reconstructions in second-order upwind schemes must be limited in regions of discontinuity or steep gradients to guarantee that no new maxima or minima are introduced. ANSYS FLUENT offers the Barth-Jespersen limiter, the Minmod limiter, and a modified Venkatakrishnan limiter, while ANSYS CFX employs the Barth-Jespersen limiter for its high resolution scheme. While the standard limiting prevents nonphysical oscillations near discontinuities, it also leads to a loss of accuracy near smooth extrema. To remedy this shortcoming, Venkatakrishnan~\cite{venkatakrishnanaccuracy} introduced a small non-vanishing differentiable parameter to prevent limiting near smooth extrema for the Barth-Jespersen limiter, while, Huynh~\cite{huynh1995accurate} employed the ratio of two adjacent second-differences of density to discern smooth regions. Sekora and Colella~\cite{sekora2009extremum} presented an approach that compared one-sided and centred estimates of the second-derivatives. If the estimates within a control volume are comparable, then the solution is smooth at the extremum; otherwise, it must be located near discontinuities, under-resolved gradients, or high-wavenumber oscillations~\cite{sekora2009extremum}. 

%To mitigate the overdissipation of vortices in second-order schemes, there are two popular approaches: vorticity confinement and artificial dissipation scaledown. In vorticity confinement, a source term is added to the momentum equation that provides an acceleration in a direction that is normal to the vorticity and its gradient. It has the effect of convecting vorticity in the opposite direction of the numerical diffusion. This approach is well suited for well-defined vortices that experience no significant distortion during the simulations. However, the approach is unsuitable for turbulent eddies that undergo severe deformations and have very short life spans. On the other hand, the artificial dissipation scaledown approach reduces the dissipation throughout the computational domain but inadvertently lowers the time step %for unsteady simulations and increases the overall computational cost.

%method employs a single scalar coefficient whose magnitude varies in the computational domain to scale down the artificial dissipation of the convective flux formulation. A major chanllenge of the method is to limit the scaledown activated region to the eddy dominated area as much as possible because further extension of the artificial dissipation reduction could lead to the slower convergence of the flow solver.


In low-speed vortical flows, due to the presence of vortices, numerous smooth extrema exist in the fields of velocity components and static pressure. Similar to the extremum-preserving limiters, L{\"o}hner~\cite{lohner2009limiters} developed a limiting algorithm to minimize the numerical dissipation of vorticity. Since the velocity typically reaches a local maximum in the tangential direction of the swirl plane, the slope limiter in second-order MUSCL schemes will switch the variable reconstruction from the first- to the zeroth-order, thereby resulting into an over-dissipation of the vortex. A helicity sensor~\cite{lohner2009limiters} is employed to identify vortical structures 
%are identified as flow regions where a helicity sensor exceeds a given threshold value. The principle direction of the vortex is defined via the vorticity vector, a normal vector toward the vortex center, and a tangential vector defined as the outer product of the vorticity and the normal vectors. 
and the van Albada limiter in the swirl plane is inactivated to reduce the numerical dissipation. Mohamed, Nadarajah and Paraschvoiu \cite{mohamed2012eddy} extended the approach and developed an eddy-preserving limiter for unsteady subsonic flows. The eddy-preserving limiter outperforms the minimal vorticity dissipation limiter~\cite{lohner2009limiters} in three primary ways. First, in~\cite{lohner2009limiters}, the vortex axis is assumed to be perpendicular to the swirl plane, which lacks generality since turbulent structures undergo stretching and bending. Instead~\cite{mohamed2012eddy} employed the enhanced swirling strength criterion of Chakraborty et al.~\cite{chakraborty2005relationships} for vortex-identification, which is based on the existence of complex eigenvalues for the velocity gradient tensor.  Second, the principle direction of eddies are determined based on eigenvectors of the velocity gradient tensor, and no specific assumption regarding their relative orientation is made. Third, in addition to inactivating the slope limiter, the dissipation is further lowered via increasing the weight of central gradients during the velocity component reconstructions on the swirl plane of the vortex. With these features, the eddy-preserving limiter is more robust and capable of dealing with complex vortical flows. 
%%%%%%%%%%%%%%%%%%%%%%%%%%%%%%%%%%%%%%%%%%%%%%%%%%%%%
The development of DES and LES as well as access to large computing infrastructures have allowed researchers to demonstrate the efficacy of these numerical frameworks on large scale hydraulic turbine flows for the past several years~\cite{jovst2011numerical, foroutan2014flow, krappel2014investigation, taheri2015detached, pacot2015prediction, wilhelm2016head}. However, a reexamination of the impact of numerical dissipation  in draft tube flows is warranted. In this paper, we extend the eddy-preserving limiter scheme by modifying the limiting algorithm for the interpolation of pressure, since a minimum pressure often exists along the axis of a free vortex. At smooth extrema of the pressure, the limiting is unnecessary and thus the conventional van Albada limiter is inactivated to reduce the dissipation. The smooth region of pressure is detected by comparing the neighbouring second difference of pressure, a criterion proposed by Huynh \cite{huynh1995accurate}. The effects of the original and extended eddy-preserving limiter scheme are initially demonstrated for a three-dimensional inviscid vortex advection and the three-dimensional convection of a vortex whose axis is parallel to the flow. We then apply the developed scheme for the simulation of the BulbT~\cite{vu2014cfd} test case.