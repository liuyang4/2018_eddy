Hydroelectricity is a source of renewable energy with numerous favourable features: safe, clean, highly efficient, low-cost, and flexible. However, the deregulation of electricity markets and the integration of wind power brought important changes to the operating patterns of hydro units. Hydro turbines have to operate at off-design conditions to adapt to the demands of the grid. At off-design operating conditions, the reliability and safety of the hydro turbines are challenged by the unsteady complex flow. Accurate numerical simulations with validation against experiments must be conducted to better understand the flow phenomenon and thus prevent the potential risks.
%%%%%%%%%%%%%%%%%%%%%%%%%%%%%%%%%%%%%%%%%%%%%%%%%%%%

%%%%%%%%%%%%%%%%%%%%%%%%%%%%%%%%%%%%%%%%%%%%%%%%%%%%
The draft tube is an important component of a hydro turbine. The role of the draft tube is to decelerate the flow from the turbine runner and convert its dynamic pressure into a rise of static pressure \cite{sick2002cfd}. At part load conditions, the energy is not fully utilized by the runner-shaft-rotor, resulting in high residual swirl when the flow exits the runner. The swirling flow with high angular momentum enters the draft tube and encounters the lower momentum fluid in the inlet conical zone. The shear layer between them gives rise to a large helical precessing vortex, commonly referred to as a vortex rope. Due to the adverse pressure gradient and hydraulic instabilities present in the draft tube, the vortex breaks down and a reverse flow occurs, further complicating the flow phenomena. As a consequence, at part load conditions, the turbine experiences severe low frequency and large amplitude pressure fluctuations induced by the vortex rope in the draft tube. The pressure fluctuations will not only cause variations in the power output, but also endangers other hydro turbine components, especially when its frequency approaches the natural frequency of the turbine \cite{dorfler2012flow}. Thus, knowledge of the dynamic load on the turbine is a major concern of the hydraulic community, and the simulation of flows in draft tubes has attracted attention during the past two decades.
%%%%%%%%%%%%%%%%%%%%%%%%%%%%%%%%%%%%%%%%%%%%%%%%%%%%%

%%%%%%%%%%%%%%%%%%%%%%%%%%%%%%%%%%%%%%%%%%%%%%%%%%%%%
In order to accurately capture the flow features, efforts have been made by employing more advanced turbulence models, improving the grid quality and extending the computational domain to the complete hydro turbine. Guidelines to choose an appropriate turbulence model often begins from past numerical simulations and {\it a priori} knowledge of inherent flow features present in the flow. Presently, access to large computational resources have enabled the research community to employ increasingly resource-demanding models; from one-equation Reynolds-averaged Navier-Stokes (RANS) closure models to detached eddy simulation (DES) and large eddy simulation (LES).
%%%%%%%%%%%%%%%%%%%%%%%%%%%%%%%%%%%%%%%%%%%%%%%%%%%%%
%%%%%%%%%%%%%%%%%%%%%%%%%%%%%%%%%%%%%%%%%%%%%%%%%%%%%
Early efforts to solve the RANS equations with the standard k-$\epsilon$ turbulence model \cite{sick2002cfd, ruprecht2002simulation} were found to be over diffusive for the simulation of draft tube flows. 
%However, the inclusion of the runner \cite{ciocan2007experimental} provided the means to resolve the vortex rope frequency but at a less desirable radius and the impact on the general performance of the draft tube such as head losses and dynamic loads were not investigated.  
Various alternative turbulence models for RANS have been applied, including the Reynolds Stress Model (RSM) \cite{sick2002cfd, stein2006numerical, jovst2009numerical, jovst2011numerical} and the k-$\omega$ SST turbulence model \cite{foroutan2012simulation, foroutan2014flow, krappel2014investigation}; however, there is no established consensus on which turbulence closure model is suitable for the capture of the vortex rope. Unlike RANS, in which all turbulence scales are modelled, LES resolves the large eddies and models the effect of small-scale turbulence through the use of subgrid-scale (SGS) models. Early LES attempts \cite{chen1995multi,skotak2000helical,guo2006large} employed coarse computational grids in the order of one to three million points that today would be classified as severely under resolved LES. Since the Reynolds number in the draft tube is on the level of $10^{6}$, and the estimated grid-point requirement for wall-resolved LES is proportional to $Re_{L_{x}}^{13/7}$, then the computational domain should contain approximately $1.39\times 10^{11}$ grid points. As expected early results \cite{chen1995multi,skotak2000helical} did not improve upon prevailing RANS simulations but Guo et al. \cite{guo2006large} captured the draft tube vortex rope frequency, resolved the axial and tangential velocity profiles downstream of the inlet, and acquired a reasonable agreement on the pressure pulsation against experimental data. The results may have been due to the use of accurate boundary conditions through the inclusion of the runner, then the accurate modelling of the turbulent flow field through the use of LES. This observation was further collaborated through an examination \cite{jovst2011numerical} of both a 23.5 million grid point LES and a 5.3 million grid point Unsteady RANS (URANS) simulation, where only the URANS simulation included a runner which, surprisingly yielded a more resolved vortex rope and frequency. Except neither produced an accurate measure of the pressure pulsation in the draft tube. Since the inclusion of the runner for wall-resolved draft tube LES simulations would be presently intractable, an hybrid RANS/LES or DES approach that employs a turbulence closure model for the RANS region and a SGS model for the LES region provides a promising avenue. DES simulations \cite{jovst2011numerical, krappel2014investigation} have been able to present the vortex rope on a wide range of test cases with various grid configurations and sizes, and flow solvers, with greater visual evidence of finer turbulent structures and well defined vortex ropes. Even so, the improved flow features have not translated to an improvement against experimental measurement of global quantities \cite{jovst2011numerical}.
%%%%%%%%%%%%%%%%%%%%%%%%%%%%%%%%%%%%%%%%%%%%%%%%%%%%%
%%%%%%%%%%%%%%%%%%%%%%%%%%%%%%%%%%%%%%%%%%%%%%%%%%%%%
Since turbulence models are not the only determinant factor of numerical simulations, a thorough investigation should include the quality of the numerical grids and the role of numerical schemes.
%%%%%%%%%%%%%%%%%%%%%%%%%%%%%%%%%%%%%%%%%%%%%%%%%%%%%

%%%%%%%%%%%%%%%%%%%%%%%%%%%%%%%%%%%%%%%%%%%%%%%%%%%%%
Initial grid studies \cite{stein2006numerical, krappel2014investigation} demonstrated an enhanced amount of turbulent structures on finer grids and in \cite{krappel2014investigation} the turbulent kinetic energy spectra showed a well represented inertial range of turbulence at both the draft tube cone and diffuser. Despite this, the investigations lacked a systematic approach typically expected for a grid convergence study and the impact on global parameters such as head losses and dynamic loads remains unresolved. Magnan et al. \cite{magnan2014challenges} conducted the most extensive grid convergence study to date and assessed the impact on several global parameters and concluded that poor grid quality severely impacts the accuracy of turbulence models, where disparate averaged axial velocities and turbulent kinetic energy contours were observed. This warrants a comprehensive investigation on the sensitivity of the resolution of the vortex rope, accurate turbulent statistics, and most importantly global parameters on the grid density and distribution within the draft tube. 
%%%%%%%%%%%%%%%%%%%%%%%%%%%%%%%%%%%%%%%%%%%%%%%%%%%%%


%%%%%%%%%%%%%%%%%%%%%%%%%%%%%%%%%%%%%%%%%%%%%%%%%%%%%
Apart from turbulence models and grid effects, discretization schemes have a tremendous impact but their role has not been carefully examined in the context of draft tube flows. The proper resolution of fine turbulent structures not only require an appropriate time-step and grid spacing but a low dissipative scheme. This can be achieved through the use of high-order methods. Higher than second-order methods are currently intractable for draft tube flows due to the large Reynolds number and computational grid, but there is room for improvement in the development of novel low dissipative second-order schemes for vortex dominated or shear flow driven flows. It is known that the dissipation error is characterized by a loss of wave amplitude, while the dispersion error is characterized by a phase difference between the numerical and analytical solution. Since the frequency of the vortex rope was well predicted while the pressure pulsation amplitudes were underestimated in most previous studies \cite{sick2002cfd, ruprecht2002simulation, stein2006numerical, jovst2009numerical}, the inherent dissipation of the numerical schemes requires a careful reexamination.


%%%%%%%%%%%%%%%%%%%%%%%%%%%%%%%%%%%%%%%%%%%%%%%%%%%%%
The eddy-preserving limiter scheme \cite{mohamed2012eddy} was designed to improve the ability of capturing vortical flow features based on the Monotonic Upstream-centered Scheme for Conservation Laws (MUSCL) scheme. The eddy-preserving limiter scheme has been implemented for turbulent flows in hydraulic turbines and some preliminary results have been presented \cite{yang2016low}. As suggested in \cite{mohamed2012eddy}, the application of a similar limiter for the interpolation of other flow variables might further enhance the vortex resolution. In this paper, we extend the eddy-preserving limiter scheme by modifying the limiting algorithm for the interpolation of pressure, since a minimum pressure often exists along the axis of a free vortex. At smooth extrema of the pressure, the limiting is unnecessary and thus the conventional van Albada limiter is inactivated to reduce the dissipation. The smooth region of pressure is detected by comparing the neighbouring second difference of pressure, a criterion proposed by Huynh \cite{huynh1995accurate}. The effects of the original and extended eddy-preserving limiter scheme are initially demonstrated for a three-dimensional inviscid vortex advection and the three-dimensional convection of a vortex whose axis is parallel to the flow. We then apply the developed scheme for the simulation of the BulbT test case.