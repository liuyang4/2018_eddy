The eddy-preserving limiter scheme has been extended by employing a new limiting algorithm for the interpolation of pressure. The conventional van Albada limiter is inactivated in the interpolation of pressure at smooth extrema. Numerical simulations of three dimensional vortex advection and the BulbT test cases were performed. Based on the comparisons, it is concluded that the original and extended eddy-preserving limiter schemes are less dissipative and capable of producing better predictions than the baseline MUSCL scheme, and the extended eddy-preserving limiter scheme has slightly improved over the original eddy-preserving limiter scheme. {\color{red} Results from the BulbT case demonstrated that the eddy-preserving limiters provided for better capture of the axial and circumferential velocities within the counter-rotating zone downstream of the hub at a lower grid resolution when compared to the baseline conventional van Albada limiter. The accurate trends are primarily influenced by the reduction of the artificial dissipation in regions of high vorticity, which in turn reduces the production of the turbulence kinetic energy. Furthermore, the extended eddy-preserving limiter provided for an improved pressure distribution.

We note that several areas related to this work remain to be explored. Inaccurate results within highly turbulent regions is sometimes faulted on the choice of turbulence model or insufficient grid resolution.  However, the results in this work allude to the fact that the numerical scheme still plays a very important role on the proper resolution of turbulent flow. We intend to further demonstrate this observation, with alternate turbulence models. In addition, the role of limiters for unsteady flows should be considered, primarily DES and LES turbulence models.}